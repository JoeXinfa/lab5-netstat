\documentclass[11pt]{article}
\usepackage[utf8]{inputenc}
\usepackage[T1]{fontenc}
\usepackage{graphicx}
\usepackage{grffile}
\usepackage{longtable}
\usepackage{wrapfig}
\usepackage{rotating}
\usepackage[normalem]{ulem}
\usepackage{amsmath}
\usepackage{textcomp}
\usepackage{amssymb}
\usepackage{capt-of}
\usepackage{hyperref}
\usepackage{minted}
\usepackage{verbatim}
\author{Student: Joseph Zhu XZ6987 \\ Professor: Mohit Tiwari \\ TA: Antonio Miguel Espinoza \\ Department of Electrical \& Computer Engineering \\ The University of Texas at Austin}
\date{\today}
\title{EEw382N Security Laboratory Exercise 5 Report}
\hypersetup{
 pdfauthor={Student: Your Name and EID Here \\ Professor: Mohit Tiwari \\ TA: Austin Harris \\ Department of Electrical \& Computer Engineering \\ The University of Texas at Austin},
 pdftitle={EEw382N Security Laboratory Exercise X Report},
 pdfkeywords={},
 pdfsubject={},
 pdfcreator={},
 pdflang={English}}
\begin{document}

\maketitle

\section{Part 1}
\label{sec:part-1}

On August 17th, I scanned using zmap on port 80 at send rate 300 packets/second
for 4 hours, and got 16204 IPs.
I did not save the command line print that has hit rate etc.
On August 18th, I scanned again with more fields (saddr, daddr, sport, dport,
success, etc). In command line print, it shows at the end:
the number of machines probed is 3176712 (send),
the number of machines responding is 11410 (recv),
and the hit rate is 0.36\%.

I wrote this code to group the IPs to networks, file \textit{findnet.py}.
Using the whois tool, one challenge is some IPs have NetRange, some have CIDR,
and some only have inetnum. Some inetnum are in range format, others in prefix.
I use one function to collect whatever whois returns for the three keys,
and another function to unify to CIDR and NetRange.
I use the netaddr library for the conversion.

The next task is to grouping. I tried the suggested method in piazza:
"not run for every IP... search through your list and group many IPs with one call..."
This may need dynamically shrink the list, taking out in-network IPs.
My implementation was looping through the IPs list (static), and searching
through networks (dynamic). I found this becomes slower when the dictionary
of the networks grows bigger.
I saved the code to my github repository and moved on to whois every IP.
Single thread would take over 15 hours for the 16204 IPs.
I used multi thread, which took 10 minutes (with 100 threads).
It is a happy surprise that I find the number of threads is not limited
to the number of cores I have in host or VM (code \textit{findnet.py}).
I saved the results to pickle for statistical analysis next.

This code \textit{netstat.py} ranks the networks by the number of IPs included.
It also gets the owner by whois the first IP and parsing \textit{OrgName},
\textit{organisation}, or \textit{descr}.
Most US IPs have well-formatted whois, while some foreign IPs do not.
I learned this when looking for network above, and again here.
The code needs to consider many cases to work with more IPs.
This may not be a surprise considering the vast amount of IPs covering
the whole earth (various nations, languages, and registeration time).

Table \ref{tbl:net20} lists the top 20 networks by the number of IPs (NIP),
along with an example IP and the owner.
As expected, the top organizations are Akamai, Cloudfare, Amazon, Aliyun,
and Google.

\begin{table}[]
\centering
\caption{\label{tbl:net20}
Top 20 networks with most IPs}
\begin{tabular}{|l|l|l|l|}
\hline
Network        & NIP & Example IP     & Owner                     \\ \hline
104.64.0.0/10  & 684 & 104.78.92.154  & Akamai Technologies, Inc. \\ \hline
23.192.0.0/11  & 406 & 23.212.206.132 & Akamai Technologies, Inc. \\ \hline
23.0.0.0/12    & 193 & 23.2.134.147   & Akamai Technologies, Inc. \\ \hline
104.16.0.0/12  & 154 & 104.17.80.50   & Cloudflare, Inc.          \\ \hline
23.72.0.0/13   & 122 & 23.75.11.119   & Akamai Technologies, Inc. \\ \hline
184.24.0.0/13  & 113 & 184.26.248.69  & Akamai Technologies, Inc. \\ \hline
52.0.0.0/11    & 101 & 52.6.25.191    & Amazon Technologies Inc.  \\ \hline
34.192.0.0/10  & 97  & 34.213.203.238 & Amazon Technologies Inc.  \\ \hline
39.96.0.0/13   & 66  & 39.105.130.108 & Aliyun Computing Co., LTD \\ \hline
18.128.0.0/9   & 60  & 18.136.65.48   & Amazon Technologies Inc.  \\ \hline
184.84.0.0/14  & 48  & 184.84.110.45  & Akamai Technologies, Inc. \\ \hline
13.224.0.0/14  & 47  & 13.227.47.22   & Amazon Technologies Inc.  \\ \hline
35.192.0.0/12  & 47  & 35.194.67.14   & Google LLC                \\ \hline
52.32.0.0/11   & 43  & 52.40.136.250  & Amazon Technologies Inc.  \\ \hline
112.124.0.0/14 & 41  & 112.127.114.74 & Aliyun Computing Co., LTD \\ \hline
3.128.0.0/9    & 41  & 3.215.10.76    & Amazon Technologies Inc.  \\ \hline
223.4.0.0/14   & 40  & 223.6.171.180  & Aliyun Computing Co., LTD \\ \hline
196.16.0.0/14  & 36  & 196.16.175.66  & ORG-IA41-AFRINIC          \\ \hline
13.32.0.0/12   & 34  & 13.33.177.55   & Amazon Technologies Inc.  \\ \hline
107.172.0.0/14 & 33  & 107.174.150.96 & ColoCrossing              \\ \hline
\end{tabular}
\end{table}


I select the Cloudflare network for some more digging,
using the command line tools we learned in the lecture
(nslookup, dig, traceroute, whois, nmap, zmap).

The nslookup result is in Appendix \ref{app:nslookup}.
The server is the DNS server my computer is querying.
The address is the DNS server and the port.
Port 53 is the reserved port for DNS.
In the answer, the first address is IPv4 and the second is IPv6.
Tool \textit{dig} returns similar information.

The traceroute result is in Appendix \ref{app:traceroute}
It went through 13 nodes between me and the server.
I ran "nmap -A" some nodes (-A to enable OS and version detection...),
but it hung, probably as the lab instruction says "it may be seen as an attack."
I ran whois on all the nodes and appended the owner behind the IPs.
Running "whois IP | grep CIDR", I find besides the un-ack 5 and 11,
Hosts 2-3 belong to the same network 172.16.0.0/12.
Hosts 7-9 belong to the same network 12.122.0.0/15.
Hosts 10-12 belong to the same network 4.0.0.0/9.
So it went through 7+ networks to reach the Cloudflare server.
I did an IP to geolocation search online, and found the server of both IPs
(198.41.215.162 and 104.17.80.50) are located in San Francisco, California.
The Cloudflare Network Map (https://www.cloudflare.com/network/)
does seem have San Francisco (even though not listed in data center).

This Cloudflare network (104.16.0.0/12) has max 2**(32-12) = 1,048,576 IPs.
I scan this network with zmap on some common ports, 60 seconds each,
and the result is shown in Table \ref{tbl:scanports}.
There are lots of machines listening on port 80/HTTP and 443/HTTPS,
but none on 20/FTP, 22/SSH, 25/SMTP.

\begin{table}[]
\centering
\caption{\label{tbl:scanports}
Cloudflare zmap scan on common ports}
\begin{tabular}{|l|l|l|l|l|l|}
\hline
Port & Protocol & send  & recv  & hitrate & IPs   \\ \hline
20   & FTP      & 12108 & 0     & 0.00\%  & 0     \\ \hline
22   & SSH      & 12509 & 0     & 0.00\%  & 0     \\ \hline
25   & SMTP     & 12484 & 0     & 0.00\%  & 0     \\ \hline
443  & HTTPS    & 17610 & 10942 & 62.14\% & 10942 \\ \hline
80   & HTTP     & 17783 & 11121 & 62.54\% & 11121 \\ \hline
\end{tabular}
\end{table}


\section{Part 2}
\label{sec:part-2}

At first I spent 2-3 hours do the 10x10 visits manually on Firefox,
then about 6 hours scripting selenium for automation (3 for Firefox, 3 for TOR).
The time investment is worthwhile because it is repeatable.
Michael Kuy and I sat together for these hours, we shared tricks/skills productively
but coded independently. My code is here \textit{openweb.py}.
Learned lessons from the manual visits, we script it the safe way.
We set the browser to clear cache after window close.
We open the browser before launching tcpdump.
We close the browser after the page loaded, then finally kill tcpdump.
I loop through the 10 sites 10 times (inner loop for sites further reduces
the risk of cache and cookies).
Running the code, I got 3x10x10=300 pcap files.
I found there is a library \textit{tbselenium} for TOR, tried it,
but still have bug, file \textit{openweb\_tor.py} as documentation.

Bash script \textit{pcap2csv.bash} prints packet information to csv files.
Python script \textit{pcapstat.py} loads, calculates statistics and plots.
Table \ref{tbl:webstat} shows the statistics of the connections.
Column "Size" is the average packet size of the 10 visits.
Column "Sent" is the average number of packets sent by my IP.
Column "Recv" is the average number of packets received by my IP.

\begin{table}[]
\centering
\caption{\label{tbl:webstat}
Statistics of the connections}
\begin{tabular}{|l|l|l|l|l|l|}
\hline
ID & Mode    & Site & Size & Sent & Recv \\ \hline
0  & firefox & cat  & 2760 & 287  & 291  \\ \hline
1  & firefox & dog  & 2936 & 286  & 293  \\ \hline
2  & firefox & egr  & 1548 & 100  & 98   \\ \hline
3  & firefox & mit  & 1957 & 232  & 228  \\ \hline
4  & firefox & unm  & 3207 & 1776 & 1752 \\ \hline
5  & firefox & cmu  & 3175 & 690  & 697  \\ \hline
6  & firefox & bkl  & 2593 & 639  & 638  \\ \hline
7  & firefox & utx  & 2194 & 455  & 459  \\ \hline
8  & firefox & asu  & 2428 & 1254 & 1278 \\ \hline
9  & firefox & utd  & 7086 & 533  & 562  \\ \hline
10 & tor     & cat  & 3135 & 177  & 212  \\ \hline
11 & tor     & dog  & 3496 & 169  & 208  \\ \hline
12 & tor     & egr  & 2199 & 79   & 88   \\ \hline
13 & tor     & mit  & 2839 & 126  & 149  \\ \hline
14 & tor     & unm  & 4995 & 1042 & 1318 \\ \hline
15 & tor     & cmu  & 4051 & 467  & 575  \\ \hline
16 & tor     & bkl  & 3518 & 413  & 495  \\ \hline
17 & tor     & utx  & 2667 & 345  & 412  \\ \hline
18 & tor     & asu  & 3217 & 885  & 1046 \\ \hline
19 & tor     & utd  & 4502 & 769  & 880  \\ \hline
20 & vpn     & cat  & 2163 & 390  & 406  \\ \hline
21 & vpn     & dog  & 2332 & 376  & 390  \\ \hline
22 & vpn     & egr  & 1299 & 177  & 182  \\ \hline
23 & vpn     & mit  & 1652 & 322  & 322  \\ \hline
24 & vpn     & unm  & 2670 & 2164 & 2141 \\ \hline
25 & vpn     & cmu  & 2610 & 858  & 880  \\ \hline
26 & vpn     & bkl  & 2238 & 760  & 756  \\ \hline
27 & vpn     & utx  & 1843 & 562  & 568  \\ \hline
28 & vpn     & asu  & 2089 & 1507 & 1517 \\ \hline
29 & vpn     & utd  & 6080 & 647  & 670  \\ \hline
\end{tabular}
\end{table}


Figure \ref{fig:aps} shows the average packet size for the 30 mode-sites.
Figure \ref{fig:npsent} shows the average number of packets sent.
Figure \ref{fig:nprecv} shows the average number of packets received.
The numbers of packets sent and received are close,
indicating most packets are ack-ed and few dropped.
We see similarities across the modes.
Looking at average packet size, egr, mit, utx are always small,
cat, asu, dog, bkl are always mid size, and
cmu, unm, utd are always at the high end.
Looking at the average number of packets sent or received,
egr, mit, dog, cat are always low,
utx, utd, bkl, cmu are in the middle, and
asu, unm are always high.
By these consistency between modes and distinction between sites,
we can determine which of the 10 websites was visited with good confidence.

Passive (aka sniffing) monitoring captures traffic from a network
by copying traffic, such as the tcpdump and wireshark we did above.
It knows which connection type we are using (regular browser, TOR, or VPN)
and if we are using (the traffic).
The benefit of TOR and VPN is encryption for higher security and privacy.
TOR encrypts the application layer including the next node destination IP address.
VPN encrypts and sends through firewall and/or web filter to VPN server.

Figure \ref{fig:ffcat0} shows the first 50
packets from tcpdump on Firefox accessing the cat site.
Figure \ref{fig:torcat0} shows the first 50
packets from tcpdump on TOR accessing the cat site.
Figure \ref{fig:vpncat0} shows the first 50
packets from tcpdump on VPN accessing the cat site.
Comparing them, I see that for regular browser, almost everything is visible to
a passive device on the network such as source/destination IPs,
source/destination ports, DNS/TCP/TLS protocols, packet size, timing, etc.
For TOR browser, DNS is hidden, TCP source/destination IPs and ports
are encrypted, TLS source IP is me but destination IP is random.
For VPN, it routes through UT network (128.83.185.41 and 129.116.67.2).
It sees DNS, but the DNS source IP is not me.
The TCP source/destination IPs and ports are visible.
The DTLS has source IP of me and destination IP of UT.


My Github repository is here. It documents my learning and testing. \\
\textit{https://github.com/JoeXinfa/lab5-netstat.git}

\section{Conclusion}
\label{sec:conclusion}
I spent 36+ hours on this lab. I like it as a combination of
network, CLI, and python.


\begin{figure}[htbp]
\centering
\includegraphics[width=.9\linewidth]{../aps.png}
\caption{\label{fig:aps}
Average packet size.
Red is for regular browser, green for TOR, and blue for VPN.}
\end{figure}

\begin{figure}[htbp]
\centering
\includegraphics[width=.9\linewidth]{../npsent.png}
\caption{\label{fig:npsent}
Average number of packets sent.
Red is for regular browser, green for TOR, and blue for VPN.}
\end{figure}

\begin{figure}[htbp]
\centering
\includegraphics[width=.9\linewidth]{../nprecv.png}
\caption{\label{fig:nprecv}
Average number of packets received.
Red is for regular browser, green for TOR, and blue for VPN.}
\end{figure}

\begin{figure}[htbp]
\centering
\includegraphics[width=1.2\linewidth]{../mon/firefox_cat_0.png}
\caption{\label{fig:ffcat0}
Packets from tcpdump on Firefox accessing the cat site}
\end{figure}

\begin{figure}[htbp]
\centering
\includegraphics[width=1.2\linewidth]{../mon/tor_cat_0.png}
\caption{\label{fig:torcat0}
Packets from tcpdump on TOR accessing the cat site}
\end{figure}

\begin{figure}[htbp]
\centering
\includegraphics[width=1.2\linewidth]{../mon/vpn_cat_0.png}
\caption{\label{fig:vpncat0}
Packets from tcpdump on VPN accessing the cat site}
\end{figure}

\appendix
\section{Appendices}
\subsection{Cloudflare nslookup}
\label{app:nslookup}

\begin{Verbatim}[obeytabs, tabsize=4]
$ nslookup cloudflare.com
Server:		127.0.0.53
Address:	127.0.0.53#53
Non-authoritative answer:
Name:	cloudflare.com
Address: 198.41.215.162
Name:	cloudflare.com
Address: 2606:4700::c629:d7a2
\end{Verbatim}

\subsection{Cloudflare traceroute}
\label{app:traceroute}

\begin{Verbatim}[obeytabs, tabsize=4]
$ sudo traceroute -I cloudflare.com
traceroute to cloudflare.com (198.41.215.162), 30 hops max, 60 byte packets
 1  10.0.2.2            Internet Assigned Numbers Authority
 2  172.20.10.1         Internet Assigned Numbers Authority
 3  172.26.96.161       Internet Assigned Numbers Authority
 4  107.72.204.92       AT&T Mobility LLC
 5  * * *
 6  12.83.186.85        AT&T Services, Inc.
 7  12.122.5.190        AT&T Services, Inc.
 8  12.122.2.197        AT&T Services, Inc.
 9  12.123.18.229       AT&T Services, Inc.
10  4.68.39.1           Level 3 Parent, LLC
11  * * *
12  4.16.234.182        Level 3 Communications, Inc.
13  198.41.215.162      Cloudflare, Inc.
\end{Verbatim}


\end{document}
